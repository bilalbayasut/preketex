\documentclass{article}
\usepackage[a4paper, margin=2.54cm]{geometry}
\usepackage{graphicx}
\usepackage{float}
\begin{document}

\section{Penentuan Capaian Pembelajaran}

\Large Mata Kuliah :

\begin{table}[H]
    \centering
    \label{penentuan-capaian-pembelajaran}
    \begin{tabular}{|l|l|l|ll}
    \cline{1-3}
    Bidang kemampuan                       & Deskripsi tingkat kemampuan                       & \begin{tabular}[c]{@{}l@{}}Deskripsi tingkat keluasan dan \\ kerumitan materi keilmuan\end{tabular}                          &  &  \\ \cline{1-3}
    Kognitif                               & Memahami                                          & \begin{tabular}[c]{@{}l@{}}Konsep pengembangan aplikasi mobile menggunakan\\ React Native.\end{tabular}                      &  &  \\ \cline{1-3}
    Psikomotorik                           & Membangun                                         & \begin{tabular}[c]{@{}l@{}}Aplikasi mobile dengan menggunakan \\ React Native\end{tabular}                                   &  &  \\ \cline{1-3}
    Afektif                                & Bekerjasama                                       & \begin{tabular}[c]{@{}l@{}}Dalam membangun aplikasi mobile \\ menggunakan React Native\end{tabular}                          &  &  \\ \cline{1-3}
    \multicolumn{3}{|l|}{\begin{tabular}[c]{@{}l@{}}Capaian Pembelajaran : Mahasiswa mampu memahami konsep React Native \\ serta mampu bekerjasama dalam membangun aplikasi mobile menggunakan \\ React Native.\end{tabular}} &  &  \\ \cline{1-3}
    \end{tabular}
    \end{table}

\section{Penetapan kemampuan akhir yang direncakanan dan indikator}
\begin{table}[H]
    \centering
    \label{my-label}
    \begin{tabular}{lll}
    \hline
    \multicolumn{1}{|l|}{Mata kuliah}          & \multicolumn{1}{l|}{:} & \multicolumn{1}{l|}{Mobile Multimedia Solution}                                                                                                                                                   \\ \hline
    \multicolumn{1}{|l|}{Capaian Pembelajaran} & \multicolumn{1}{l|}{:} & \multicolumn{1}{l|}{\begin{tabular}[c]{@{}l@{}}Mahasiswa mampu memahami konsep React Native \\ serta mampu bekerjasama dalam membangun \\ aplikasi mobile menggunakan React Native.\end{tabular}} \\ \hline
                                               &                        &                                                                                                                                                                                                  
    \end{tabular}
    \end{table}
\end{document}