\documentclass{article}
\usepackage[a4paper, margin=2.54cm]{geometry}
\usepackage{graphicx}
\usepackage{enumitem}
\usepackage{float}
\usepackage{array,etoolbox}
\usepackage{booktabs}

\newcommand{\tabitem}{~~\llap{\textbullet}~~}

\preto\tabular{\setcounter{magicrownumbers}{0}}
\newcounter{magicrownumbers}
\newcommand\rownumber{\stepcounter{magicrownumbers}\arabic{magicrownumbers}}
\begin{document}

\section{Penentuan Capaian Pembelajaran}

\Large Mata Kuliah :

\begin{table}[H]
    \centering
    \label{penentuan-capaian-pembelajaran}
    \begin{tabular}{|l|l|l|ll}
    \cline{1-3}
    Bidang kemampuan                       & Deskripsi tingkat kemampuan                       & \begin{tabular}[c]{@{}l@{}}Deskripsi tingkat keluasan dan \\ kerumitan materi keilmuan\end{tabular}                          &  &  \\ \cline{1-3}
    Kognitif                               & Memahami                                          & \begin{tabular}[c]{@{}l@{}}Konsep pengembangan aplikasi mobile menggunakan\\ React Native.\end{tabular}                      &  &  \\ \cline{1-3}
    Psikomotorik                           & Membangun                                         & \begin{tabular}[c]{@{}l@{}}Aplikasi mobile dengan menggunakan \\ React Native\end{tabular}                                   &  &  \\ \cline{1-3}
    Afektif                                & Bekerjasama                                       & \begin{tabular}[c]{@{}l@{}}Dalam membangun aplikasi mobile \\ menggunakan React Native\end{tabular}                          &  &  \\ \cline{1-3}
    \multicolumn{3}{|l|}{\begin{tabular}[c]{@{}l@{}}Capaian Pembelajaran : Mahasiswa mampu memahami konsep React Native \\ serta mampu bekerjasama dalam membangun aplikasi mobile menggunakan \\ React Native.\end{tabular}} &  &  \\ \cline{1-3}
    \end{tabular}
    \end{table}

\section{Penetapan kemampuan akhir yang direncakanan dan indikator}
\begin{table}[H]
    \centering
    \begin{tabular}{lll}
    \hline
    \multicolumn{1}{|l|}{Mata kuliah}          & \multicolumn{1}{l|}{:} & \multicolumn{1}{l|}{Mobile Multimedia Solution}                                                                                                                                                   \\ \hline
    \multicolumn{1}{|l|}{Capaian Pembelajaran} & \multicolumn{1}{l|}{:} & \multicolumn{1}{l|}{\begin{tabular}[c]{@{}l@{}}Mahasiswa mampu memahami konsep React Native \\ serta mampu bekerjasama dalam membangun \\ aplikasi mobile menggunakan React Native.\end{tabular}} \\ \hline
                                               &                        &                                                                                                                                                                                                  
    \end{tabular}
    \end{table}


    % \begin{center}
    %     \begin{tabular}{ |l|l|l|l| } 
    %      \hline
    %      No & Kemampuan akhir yang direncanakan & Indikator & Materi\\
    %      \hline
    %      1 & cell5 & 
    %      \begin{enumerate}
    %         \item This is item 1
    %         \item This is item 2
    %      \end{enumerate} 
    %      & materi \\ 

    %      \hline
    %     \end{tabular}
    %     \end{center}

    \begin{center}
    \begin{tabular}{|l|l|l|l|}
        \hline
        No & \begin{minipage}{3in}Kemampuan akhir \\
            yang direncanakan\end{minipage} & 
            Indikator & Materi\\
       \hline
        1 & 
        \begin{minipage}{3in}
          \vskip 4pt
          \begin{enumerate}
         \item Mahasiswa mampu menjelaskan konsep dasar React Native
         \item Mahasiswa mampu menjelaskan alasan kenapa menggunakan React Native
         \end{enumerate}
         \vskip 4pt
       \end{minipage}
       & test 
       & test
       \\
        \hline
       \end{tabular}
    \end{center}

    % \begin{table}[H]
    %     \centering
    %     \begin{tabular}{|l|l|l|l|}
    %     \hline
    %     No & \begin{tabular}[c]{@{}l@{}}Kemampuan akhir\\ yang direncanakan\end{tabular}                     & Indikator & Materi       \\ \hline
    %     1  & 
    %     \begin{tabular}[c]{@{}l@{}}
    %     Mahasiswa dapat menjelaskan\\ konsep dasar React Native
    %     \end{tabular} 
    %     & Mahasiswa 
    %     & Introduction \\ \hline
    %        &                                                                                                 &           &              \\ \hline
    %     \end{tabular}
    %     \end{table}
        

% \begin{enumerate}
            %     \label{indikator}
            %     \item Topic
            %     \begin{enumerate}[label*=\arabic*.]
            %       \item First Subtopic
            %       \item Second Subtopic
            %     \end{enumerate}
            %   \end{enumerate}

            % \begin{tabular}{|p{0.4\textwidth}|p{.03\textwidth}|p{0.4\textwidth}|}
            %     %%%
            %     \begin{itemize}
            %       \item Sample Text
            %       \item Sample Text
            %       \end{itemize}
            %       & Hi &
            %       \begin{itemize}
            %       \item Sample Text
            %       \item Sample Text
            %     \end{itemize}
            %     %%%
                
            %     \end{tabular}
\end{document}