\documentclass{article}
\usepackage[a4paper, margin=2.54cm]{geometry}
\usepackage{graphicx}

\begin{document}

\title{Analisis Instruksional}
\author{Bilal Luqman Bayasut BCS, MCS}

\begin{center}
\huge ANALISIS INSTRUKSIONAL
\end{center}
\vfill
\begin{figure}[h!]
    \begin{center}
        \includegraphics[scale=0.1]{sources/Logo_Kemenristekdikti.png}
        \label{fig:Logo_Kemenristekdikti}
    \end{center}
\end{figure}
\vfill
\begin{center}
    \huge Oleh : \\
        \huge Bilal Luqman Bayasut BCS, MCS \\
        \huge Universitas Wijaya Putra
\end{center}
\vfill
\begin{center}
        \huge KEMENTRIAN RISET TEKNOLOGI PENDIDIKAN TINGGI \\
        \huge KOORDINASI PERGURUAN TINGGI SWASTA \\
        \huge WILAYAH VII \\
        \huge 2017
\end{center}

\newpage
\begin{table}[h!]
    
    \label{tab:table1}
    \begin{tabular}{l l l}
      Nama Mata Kuliah & : & Konsep Bahasa Pemograman\\
      Kode Mata Kuliah & : & 53B001\\
      Bobot & : & 3 (tiga)  S K S\\
      Semester & : & 1 (Satu)\\
      Fakultas / Program Studi & : & Ts\\
      Dosen Pengampu/ NIDN & : & Ts\\
      Diskripsi Mata Kuliah & : & Ts\\
    \end{tabular}
    \\[50px]
    \begin{tabular}{cllll}
        Pertemuan & Standar Kompetensi (SK) & Kompetensi Akademik (KA) &  &  \\
        1         & satu                    & dua                      &  &  \\
                  &                         &                          &  &  \\
                  &                         &                          &  & 
        \end{tabular}
  \end{table}

\end{document}